\documentclass[]{article}
\usepackage{graphicx}
\usepackage{placeins}
%opening
\title{Software Requirement Specification Document}
\author{Sandra Fares,Nour Ahmed ,Samiha Hesham, Mariam Hesham }

\begin{document}

\maketitle

\section{Introduction}

\subsection{Purpose of this document}
The main purpose of this document is to clarify and demonstrate the requirements of a Clinic System. These requirements include booking appointments, contacting Doctors or Physician assistants, and viewing patients' medical information. Our aim is to assist patients' to reserve appointments and reach out to healthcare professionals easily. Moreover, our system shall make it easier for Doctors to view patients' medical information, in order to provide accurate diagnosis. This documentation shall provide an explanation of each phase, along with an illustration on how this system is going to function. 

\subsection{ Scope of this document}
The scope of this document is to break down our client's problems into chunks that can be solved more readily.It also helps to increase understanding of issues and makes them easier to be tackled through our system and tested properly to fulfil the optimum solutions. 


\subsection{Overview}

The project clinic management is developed to facilitate the communication process between the doctor, the receptionist, the assistant and the patient. The software would be managed by two admins one is doctor and the other is receptionist. The doctor can view patient details and after checking up the patient, the recommended medicines for the patient are fed into the database by the doctor and are sent to receptionist. The patient can book an appointment and state the medical history .Receptionist would be responsible for viewing appointments and save it in the database along with their details .The receptionist can then generate bill and feed into the database. The system also maintains patient’s history so that doctor or assistant can view them anytime .The assistant would be responsible for booking an operation and follow up with the patient after the surgery .The system can thus reduce complexity in maintaining patient’s records.

\subsection{Business Context}

The project aims to build an application program to reduce the manual work for managing the clinic. It helps to end the complexities involved in accounting along with detailed profit and loss reports .The system can provide convenient access to information and making it less likely that mistakes would be made due to illegible writing. The internet-based access improves the ability to remotely access the data and making it easier to maintain and quicker to search through for relevant. The system also saves time for medical staff to focus on more important tasks at hand and reduces wait times which is otherwise spent filling out forms in and reduces the waiting time for patients at the clinics.


\section{General description}
\subsection{product functions}

\begin{enumerate}
  \item The system must be fully dynamic,in which the admin is fully controlling the system
  \item The users(patients,doctor,receptionists,assistants) must be able to control their information 
  \item The patient must be able to reserve his/her appointment online and choose either paying online or cash.
  \item The system should be able to send automatic emails and SMS to the patients.
  \item The admin should have full authority to control all staff permissions.
  \item The doctor must be able to see full information of the patient.
  \item The doctor should has the ability to write a report to the patient's case with all details.
  \item The doctor must be able to reserve operation for the patient if needed.
  \item The assistant must organize the doctor's schedule and add,edit or delete operations to it.
  \item The assistant should perform a broadcast en-forming patients of any cancellation in the doctor's schedule.
  \item The patients must have the authority to edit their appointment details.
  \item The receptionist must view patients' reservations and should deal with financial staff.
\end{enumerate}

\subsection{Similar System Information}
This software is similar to various generic products. However, MedDNA and eClinic are the most similar to our system. These systems are stand-alone clinic management systems that are available for installation by any customer in need. Their purpose is to help in managing appointments, viewing patients' history, and offering effortless communication between the Doctors and their Patients, just like our system is intended to do.


\subsection{ User Characteristics}
The admin, doctor, receptionists, assistants and patients will be the main users. The system is also designed to be user-friendly. 

\begin{itemize}
  \item Admin
  \item Receptionists
  \item Doctor
  \item Assistants
  \item Patients
  
  
\end{itemize}

Admin: Admin should have prior knowledge of the system. Admin is able to control the whole system. He/she can add, delete, update and modify the system.


Receptionists: in order view the details of the patients come for the treatment and accordingly provides identity to them also, schedule the appointments of the day and deals with payment methods. 


Doctor: Doctor should fairly know about the usage of the system. Doctor are able to see the respective appointments taken. And also, can view patient’s details and records and set for them operations if needed.


Assistants: they are mainly managing the chatting system for the clinic with patients and they must  view, edit ,delete and add operations in the doctor’s schedule.


Casual users: Anyone can view the information of the polyclinic. Patients can view their own records and doctor’s details and timings to take appointment online.


\subsection{ User Problem Statement}
Our client needs to replace all the manual operations done daily in his clinic into automated ones.Starting from patient's booking their appointments,saving their history to be easier to retrieve on the follow up day to booking surgeries too .Also they need a chatting system in order to facilitate patients contact with our client or his assistant .All of which should be easy to use , without any complexities to provide fast interactions without consuming time while dealing with our software.

\section{Functional Requirements}

\FloatBarrier
\begin{table}[h]
\caption{ }
\label{tab:my-table}
\begin{tabular}{|p{0.25\textwidth}|p{0.75\textwidth}|}
\hline
\textbf{Function:} & Check email validation.
\\ \hline
\textbf{ID}  &      FR00      

\\ \hline
\textbf{Description}    &    This Function is to check if the email is valid before inserting it in the database.                                                                 
\\ \hline
\textbf{Type}    &         Boolean

\\ \hline
\textbf{Input}        & Email


\\ \hline
\textbf{Action}            & Checks if the email entered in the text field  is in the proper format and  if the email is valid  it returns true means that the email is valid to be stored if exists in the database.

\\ \hline
\textbf{Output}            & Boolean if email is valid

\\ \hline
\textbf{Precondition}           &   User entered email into the required text field.

\\ \hline
\textbf{Post condition}          & return to login function (FR03)  


\\ \hline
\textbf{Dependncies}           & return to login function (FR03).
\\ \hline
\end{tabular}
\end{table}

\FloatBarrier
\begin{table}[h]
\caption{ }
\label{tab:my-table}
\begin{tabular}{|p{0.25\textwidth}|p{0.75\textwidth}|}
\hline
\textbf{Function:} & Check password validation
\\ \hline
\textbf{ID}  &     FR01       

\\ \hline
\textbf{Description}    &      This Function is to check if the password is valid before inserting it in the database.                                                              
\\ \hline
\textbf{Type}    &         Boolean

\\ \hline
\textbf{Input}        & Password


\\ \hline
\textbf{Action}            & Checks if the password is valid in format. If yes it returns true means that the password is valid to be stored if exist in the database with its email.

\\ \hline
\textbf{Output}            & Boolean if password valid

\\ \hline
\textbf{Precondition}           &   User entered password into the required text field

\\ \hline
\textbf{Post condition}           & Return to login function (FR03)


\\ \hline
\textbf{Dependncies}           & FR03
\\ \hline
\end{tabular}
\end{table}

\FloatBarrier
\begin{table}[h]
\caption{ }
\label{tab:my-table}
\begin{tabular}{|p{0.25\textwidth}|p{0.75\textwidth}|}
\hline
\textbf{Function:} & Check phone Number validation
\\ \hline
\textbf{ID}  &            FR02

\\ \hline
\textbf{Description}    &     This Function is to check if the phone number is valid before storing it in the database.                                                                
\\ \hline
\textbf{Type}    &       Boolean  

\\ \hline
\textbf{Input}        & Phone number


\\ \hline
\textbf{Action}            & Checks if the phone number entered in the  field  is in the proper format and  if it is valid in it's format it returns true means that it is valid to be stored in the database.

\\ \hline
\textbf{Output}            & Boolean if phone number is valid.

\\ \hline
\textbf{Precondition}           &   User enter the phone number into the required text field.

\\ \hline
\textbf{Post condition}           & Return to login function (FR03)


\\ \hline
\textbf{Dependncies}           & FR03
\\ \hline
\end{tabular}
\end{table}

\FloatBarrier
\begin{table}[h]
\caption{ }
\label{tab:my-table}
\begin{tabular}{|p{0.25\textwidth}|p{0.75\textwidth}|}
\hline
\textbf{Function:} & Login.
\\ \hline
\textbf{ID}  &    FR03        

\\ \hline
\textbf{Description}    &       This Function is for the user to login into the system using his/her account.                                                              
\\ \hline
\textbf{Type}    &         Boolean

\\ \hline
\textbf{Input}        & Email and Password


\\ \hline
\textbf{Action}            & Checks if the data are filled and compare the record that was entered before to the record in the database, The function returns true otherwise returns false.

\\ \hline
\textbf{Output}            & Boolean acceptance login.

\\ \hline
\textbf{Precondition}           &   The user needs to into his email and password into the Text fields and that the inputs are verified (FR00, FR01)

\\ \hline
\textbf{Post condition}           & Redirect to the home page


\\ \hline
\textbf{Dependncies}           & FR00, FR01
\\ \hline
\end{tabular}
\end{table}

\FloatBarrier
\begin{table}[h]
\caption{ }
\label{tab:my-table}
\begin{tabular}{|p{0.25\textwidth}|p{0.75\textwidth}|}
\hline
\textbf{Function:} & Forget Password
\\ \hline
\textbf{ID}  &            FR04

\\ \hline
\textbf{Description}    &       This Function exists if the user forgets the password, it generates a new password and sends email with it to help him login successfully again.                                                               
\\ \hline
\textbf{Type}    &       Boolean  

\\ \hline
\textbf{Input}        & Email


\\ \hline
\textbf{Action}            & Firstly the email is checked using FR00 then if the email exists in the database, the function sends email FR14 with the new password generated.

\\ \hline
\textbf{Output}            & Record updated

\\ \hline
\textbf{Pre condition}           &   The user enters his email in the Text field.

\\ \hline
\textbf{Post condition}     & Notification is sent to the user with new password.


\\ \hline
\textbf{Dependncies}           & FR00, FR14
\\ \hline
\end{tabular}
\end{table}


\FloatBarrier
\begin{table}[h]
\caption{Class Name - }
\label{tab:my-table}
\begin{tabular}{|p{0.25\textwidth}|p{0.75\textwidth}|}
\hline
\textbf{Function:} & Create User.
\\ \hline
\textbf{ID}  & FR05          
\\ \hline
\textbf{Description}    &  The functions get the user type ID according from which page the inserting process whether the patient's registration page or from the administrator is happening, then it
uses the input data to insert a new row in users table in
the system database                                                              \\ \hline
\textbf{Type}    & void     
\\ \hline
\textbf{Input}  & user type id and user data
\\ \hline
\textbf{Action} &  checks if the user's email,password and phone number are validated, if so it enters the user record in the database and returns true else it returns false and an error message will appear 
\\ \hline
\textbf{Output} & New User Record in Database
\\ \hline
\textbf{Pre condition}  & User's information is right and validated     
\\ \hline
\textbf{Post condition} & Creating a new row in database table with new ID for the user
\\ \hline
\textbf{Dependncies} & FR00 , FR01 , FR02  
\\ \hline
\end{tabular}
\end{table}

\FloatBarrier
\begin{table}[h]
\caption{Class Name - }
\label{tab:my-table}
\begin{tabular}{|p{0.25\textwidth}|p{0.75\textwidth}|}
\hline
\textbf{Function:} & Delete User.
\\ \hline
\textbf{ID}        &FR06            
\\ \hline
\textbf{Description} & The Function is to remove the selected user row from users table                                                         \\ \hline
\textbf{Type}     & Boolean       
\\ \hline
\textbf{Input}   & ID of the user selected to be removed 
\\ \hline
\textbf{Action}  & check if the user id exists in database
\\ \hline
\textbf{Output}  & A Success Message if User is deleted 
\\ \hline
\textbf{Pre condition}  & Administrator chooses which user to be deleted  
\\ \hline
\textbf{Post condition} & User Record is deleted from database and database is updated 
\\ \hline
\textbf{Dependncies}  &FR05 
\\ \hline
\end{tabular}
\end{table}

\FloatBarrier
\begin{table}[h]
\caption{Class Name - }
\label{tab:my-table}
\begin{tabular}{|p{0.25\textwidth}|p{0.75\textwidth}|}
\hline
\textbf{Function:} & Search User.
\\ \hline
\textbf{ID}  & FR07           
\\ \hline
\textbf{Description} & The Function returns the user with the same id or name that being searched for by one of our staff users                                              
\\ \hline
\textbf{Type}  & Return Function        

\\ \hline
\textbf{Input}  & Id or Name of the desired user 

\\ \hline
\textbf{Action} & check if user id or name matches any of records in database , if there is a user show it else returns null
\\ \hline
\textbf{Output} & if Users details exists
\\ \hline
\textbf{Pre condition}   &   One of staff users enters users id or name needed to be found
\\ \hline
\textbf{Post condition} & user record is retrieved from database if is found 
\\ \hline
\textbf{Dependncies} & -- 
\\ \hline
\end{tabular}
\end{table}

\FloatBarrier
\begin{table}[h]
\caption{Class Name - }
\label{tab:my-table}
\begin{tabular}{|p{0.25\textwidth}|p{0.75\textwidth}|}
\hline
\textbf{Function:} & Update User
\\ \hline
\textbf{ID}  &  FR08          
\\ \hline
\textbf{Description} & The Function performs an sql query to users table to update a certain record according to a specific user ID                                                     
\\ \hline
\textbf{Type}    &  Boolean       

\\ \hline
\textbf{Input}  & User ID and the desired category to be updated with it's new record  
\\ \hline
\textbf{Action} & check if those data can be updated 
\\ \hline
\textbf{Output}  & success message or failure message  
\\ \hline
\textbf{Pre condition}  &  check if user exists in database and has the right to update 

\\ \hline
\textbf{Post condition} & records in database are updated  
\\ \hline
\textbf{Dependncies}    & -- 
\\ \hline
\end{tabular}
\end{table}

\FloatBarrier
\begin{table}[h]
\caption{Class Name - }
\label{tab:my-table}
\begin{tabular}{|p{0.25\textwidth}|p{0.75\textwidth}|}
\hline
\textbf{Function:} & Create Usertype
\\ \hline
\textbf{ID}  & FR09           

\\ \hline
\textbf{Description}  & this function is to add new user type to the system in the database                                                                     
\\ \hline
\textbf{Type}   & Boolean     

\\ \hline
\textbf{Input}  & User Type name

\\ \hline
\textbf{Action}  & Check if the user type name is not duplicated in the database , if so it enters the user type record in the database and returns true else return false

\\ \hline
\textbf{Output}  & Success Message if user type is created 

\\ \hline
\textbf{Pre condition}  & Check if user type doesn't exist   
\\ \hline
\textbf{Post condition}  & Creating a new database record with a new ID and add the new user type name in it 
\\ \hline
\textbf{Dependncies} & ---
\\ \hline
\end{tabular}
\end{table}
\FloatBarrier
\begin{table}[h]
\caption{Class Name - }
\label{tab:my-table}
\begin{tabular}{|p{0.25\textwidth}|p{0.75\textwidth}|}
\hline
\textbf{Function:} & Update Usertype
\\ \hline
\textbf{ID}  & FR10          

\\ \hline
\textbf{Description} & this function is to Update an existing user type in database according to the user type ID \\ \hline
\textbf{Type}    &  Boolean       

\\ \hline
\textbf{Input}   & The User type data to be updated 

\\ \hline
\textbf{Action}  & Check if user type is updated and return true else return false or error message 

\\ \hline
\textbf{Output}  & Success Message if updated

\\ \hline
\textbf{Pre condition}  & Check if user type exists    

\\ \hline
\textbf{Post condition} & Update the system database row of the user type according to the user type ID  
\\ \hline
\textbf{Dependncies}  & FR09
\\ \hline
\end{tabular}
\end{table}

\FloatBarrier
\begin{table}[h]
\caption{Class Name - }
\label{tab:my-table}
\begin{tabular}{|p{0.25\textwidth}|p{0.75\textwidth}|}
\hline
\textbf{Function:} & Delete UserType 
\\ \hline
\textbf{ID}  & FR11          

\\ \hline
\textbf{Description}  & This function is to delete the selected user type row from user type in database                                                                     
\\ \hline
\textbf{Type}   & Boolean         

\\ \hline
\textbf{Input}  & ID of which user type to be deleted 

\\ \hline
\textbf{Action} & Check if user Type id exists in the database to delete it 

\\ \hline
\textbf{Output}  & Message if user type deleted  

\\ \hline
\textbf{Pre condition}   & The selected user type needed to be deleted   

\\ \hline
\textbf{Post condition}  & Updating the database by removing the user type id row from table 

\\ \hline
\textbf{Dependncies}    & ---- 
\\ \hline
\end{tabular}
\end{table}

\FloatBarrier
\begin{table}[h]
\caption{Class Name - }
\label{tab:my-table}
\begin{tabular}{|p{0.25\textwidth}|p{0.75\textwidth}|}
\hline
\textbf{Function:} & Indicates whether this class is abstract or concrete.
\\ \hline
\textbf{ID}  &            

\\ \hline
\textbf{Description}    &                                                                     
\\ \hline
\textbf{Type}    &         

\\ \hline
\textbf{Input}        & 


\\ \hline
\textbf{Action}            & 

\\ \hline
\textbf{Output}            & 

\\ \hline
\textbf{Pre condition}           &   

\\ \hline
\textbf{Post condition}           & 


\\ \hline
\textbf{Dependncies}           & 
\\ \hline
\end{tabular}
\end{table}

\FloatBarrier
\begin{table}[h]
\caption{Class Name - }
\label{tab:my-table}
\begin{tabular}{|p{0.25\textwidth}|p{0.75\textwidth}|}
\hline
\textbf{Function:} & Indicates whether this class is abstract or concrete.
\\ \hline
\textbf{ID}  &            

\\ \hline
\textbf{Description}    &                                                                     
\\ \hline
\textbf{Type}    &         

\\ \hline
\textbf{Input}        & 


\\ \hline
\textbf{Action}            & 

\\ \hline
\textbf{Output}            & 

\\ \hline
\textbf{Pre condition}           &   

\\ \hline
\textbf{Post condition}           & 


\\ \hline
\textbf{Dependncies}           & 
\\ \hline
\end{tabular}
\end{table}

\FloatBarrier
\begin{table}[h]
\caption{Class Name - }
\label{tab:my-table}
\begin{tabular}{|p{0.25\textwidth}|p{0.75\textwidth}|}
\hline
\textbf{Function:} & Indicates whether this class is abstract or concrete.
\\ \hline
\textbf{ID}  &            

\\ \hline
\textbf{Description}    &                                                                     
\\ \hline
\textbf{Type}    &         

\\ \hline
\textbf{Input}        & 


\\ \hline
\textbf{Action}            & 

\\ \hline
\textbf{Output}            & 

\\ \hline
\textbf{Pre condition}           &   

\\ \hline
\textbf{Post condition}           & 


\\ \hline
\textbf{Dependncies}           & 
\\ \hline
\end{tabular}
\end{table}

\FloatBarrier
\begin{table}[h]
\caption{Class Name - }
\label{tab:my-table}
\begin{tabular}{|p{0.25\textwidth}|p{0.75\textwidth}|}
\hline
\textbf{Function:} & Indicates whether this class is abstract or concrete.
\\ \hline
\textbf{ID}  &            

\\ \hline
\textbf{Description}    &                                                                     
\\ \hline
\textbf{Type}    &         

\\ \hline
\textbf{Input}        & 


\\ \hline
\textbf{Action}            & 

\\ \hline
\textbf{Output}            & 

\\ \hline
\textbf{Pre condition}           &   

\\ \hline
\textbf{Post condition}           & 


\\ \hline
\textbf{Dependncies}           & 
\\ \hline
\end{tabular}
\end{table}

\FloatBarrier
\begin{table}[h]
\caption{Class Name - }
\label{tab:my-table}
\begin{tabular}{|p{0.25\textwidth}|p{0.75\textwidth}|}
\hline
\textbf{Function:} & Indicates whether this class is abstract or concrete.
\\ \hline
\textbf{ID}  &            

\\ \hline
\textbf{Description}    &                                                                     
\\ \hline
\textbf{Type}    &         

\\ \hline
\textbf{Input}        & 


\\ \hline
\textbf{Action}            & 

\\ \hline
\textbf{Output}            & 

\\ \hline
\textbf{Pre condition}           &   

\\ \hline
\textbf{Post condition}           & 


\\ \hline
\textbf{Dependncies}           & 
\\ \hline
\end{tabular}
\end{table}

\FloatBarrier
\begin{table}[h]
\caption{Class Name - }
\label{tab:my-table}
\begin{tabular}{|p{0.25\textwidth}|p{0.75\textwidth}|}
\hline
\textbf{Function:} & Indicates whether this class is abstract or concrete.
\\ \hline
\textbf{ID}  &            

\\ \hline
\textbf{Description}    &                                                                     
\\ \hline
\textbf{Type}    &         

\\ \hline
\textbf{Input}        & 


\\ \hline
\textbf{Action}            & 

\\ \hline
\textbf{Output}            & 

\\ \hline
\textbf{Pre condition}           &   

\\ \hline
\textbf{Post condition}           & 


\\ \hline
\textbf{Dependncies}           & 
\\ \hline
\end{tabular}
\end{table}

\FloatBarrier
\begin{table}[h]
\caption{Class Name - }
\label{tab:my-table}
\begin{tabular}{|p{0.25\textwidth}|p{0.75\textwidth}|}
\hline
\textbf{Function:} & Indicates whether this class is abstract or concrete.
\\ \hline
\textbf{ID}  &            

\\ \hline
\textbf{Description}    &                                                                     
\\ \hline
\textbf{Type}    &         

\\ \hline
\textbf{Input}        & 


\\ \hline
\textbf{Action}            & 

\\ \hline
\textbf{Output}            & 

\\ \hline
\textbf{Pre condition}           &   

\\ \hline
\textbf{Post condition}           & 


\\ \hline
\textbf{Dependncies}           & 
\\ \hline
\end{tabular}
\end{table}

\FloatBarrier
\begin{table}[h]
\caption{Class Name - }
\label{tab:my-table}
\begin{tabular}{|p{0.25\textwidth}|p{0.75\textwidth}|}
\hline
\textbf{Function:} & Indicates whether this class is abstract or concrete.
\\ \hline
\textbf{ID}  &            

\\ \hline
\textbf{Description}    &                                                                     
\\ \hline
\textbf{Type}    &         

\\ \hline
\textbf{Input}        & 


\\ \hline
\textbf{Action}            & 

\\ \hline
\textbf{Output}            & 

\\ \hline
\textbf{Pre condition}           &   

\\ \hline
\textbf{Post condition}           & 


\\ \hline
\textbf{Dependncies}           & 
\\ \hline
\end{tabular}
\end{table}

\FloatBarrier
\begin{table}[h]
\caption{Class Name - }
\label{tab:my-table}
\begin{tabular}{|p{0.25\textwidth}|p{0.75\textwidth}|}
\hline
\textbf{Function:} & Indicates whether this class is abstract or concrete.
\\ \hline
\textbf{ID}  &            

\\ \hline
\textbf{Description}    &                                                                     
\\ \hline
\textbf{Type}    &         

\\ \hline
\textbf{Input}        & 


\\ \hline
\textbf{Action}            & 

\\ \hline
\textbf{Output}            & 

\\ \hline
\textbf{Pre condition}           &   

\\ \hline
\textbf{Post condition}           & 


\\ \hline
\textbf{Dependncies}           & 
\\ \hline
\end{tabular}
\end{table}

\FloatBarrier
\begin{table}[h]
\caption{Class Name - }
\label{tab:my-table}
\begin{tabular}{|p{0.25\textwidth}|p{0.75\textwidth}|}
\hline
\textbf{Function:} & Indicates whether this class is abstract or concrete.
\\ \hline
\textbf{ID}  &            

\\ \hline
\textbf{Description}    &                                                                     
\\ \hline
\textbf{Type}    &         

\\ \hline
\textbf{Input}        & 


\\ \hline
\textbf{Action}            & 

\\ \hline
\textbf{Output}            & 

\\ \hline
\textbf{Pre condition}           &   

\\ \hline
\textbf{Post condition}           & 


\\ \hline
\textbf{Dependncies}           & 
\\ \hline
\end{tabular}
\end{table}

\FloatBarrier
\begin{table}[h]
\caption{Class Name - }
\label{tab:my-table}
\begin{tabular}{|p{0.25\textwidth}|p{0.75\textwidth}|}
\hline
\textbf{Function:} & Indicates whether this class is abstract or concrete.
\\ \hline
\textbf{ID}  &            

\\ \hline
\textbf{Description}    &                                                                     
\\ \hline
\textbf{Type}    &         

\\ \hline
\textbf{Input}        & 


\\ \hline
\textbf{Action}            & 

\\ \hline
\textbf{Output}            & 

\\ \hline
\textbf{Pre condition}           &   

\\ \hline
\textbf{Post condition}           & 


\\ \hline
\textbf{Dependncies}           & 
\\ \hline
\end{tabular}
\end{table}



\FloatBarrier
\section{Interface Requirements}
This section describes how the software interfaces with other software products or users for input or output. Examples of such interfaces include library routines, token streams, shared memory, data streams, and so forth. 

\subsection{User Interfaces}
Use some software for primitive plan of your project.
Describes how this product interfaces with the user. 

\subsubsection {GUI}
Describes the graphical user interface if present. This section should include a set of screen dumps or mockups to illustrate user interface features. 
If the system is menu-driven, a description of all menus and their components should be provided. 

\subsubsection { CLI}
N/A

\subsection {API}
N/A


\section{Design Constraints}
Specifies any constraints for the design team using this document. 

\section{Other non-functional attributes}


\subsection {Security}
This system is provided with authentication without which no user can pass. So only the legitimate users are allowed to use the application. If the legitimate user’s share the authentication information then the system is open to outsiders.



\subsection {Reliability}
Good validations of user inputs will be done to avoid incorrect storage of records.
\subsection {Maintainability}



\subsection {Portability}
This system can be installed in any personal computers supporting windows operating system platform.
\subsection {Flexibility}
The system keeps on updating the data according to the transactions that takes place.

\section{Preliminary Object-Oriented Domain Analysis}
This section presents a list of the fundamental objects that must be modeled within the system to satisfy its requirements. The purpose is to provide an alternative, ''structural'' view on the requirements stated above and how they might be satisfied in the system. A primitive class diagram to be delivered.

\subsection{Class diagram}
This section should contain a set of graphs that illustrate the primary inheritance hierarchy (is-kind-of) for the system. For example: 



\subsection{ER diagram}
This section presents a more detailed description of each class identified during the OO Domain Analysis.
Each class description should conform to the following structure: 



\section{Operational Scenarios}
use case diagram

\section{Project Plan}
This section provides an initial version of the project plan, including the major tasks to be accomplished, their interdependencies, and their tentative start/stop dates. The plan also includes information on hardware, software, and  resource requirements. 
The project plan should be accompanied by one or more PERT or GANTT charts. 

\section{Appendices}


\subsection{ Abbreviations}
\begin{itemize}

\item SMS = Short Message Service
\item ER=Entity Relationship diagram

\end{itemize}

\subsection{Collected material}

\section {References}

\bibliographystyle{IEEEtranS}

\bibliography{cite}

\end{document}
