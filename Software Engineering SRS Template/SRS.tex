\documentclass[]{article}
\usepackage{graphicx}
%opening
\title{Software Requirement Specification Document}
\author{Mariam Hesham,Nour Ahmed ,Samiha Hesham, Sandra Fares }

\begin{document}

\maketitle

\section{Introduction}

\subsection{Purpose of this document}
The main purpose of this document is to clarify and demonstrate the requirements of a Clinic System. These requirements include booking appointments, contacting Doctors or Physician assistants, and viewing patients' medical information. Our aim is to assist patients' to reserve appointments and reach out to healthcare professionals easily. Moreover, our system shall make it easier for Doctors to view patients' medical information, in order to provide accurate diagnosis. This documentation shall provide an explanation of each phase, along with an illustration on how this system is going to function. 

\subsection{ Scope of this document}
With the increase in the number of patients visiting the clinic , it has become difficult to manage the appointments, retrieving patient's medical history and sometimes contacting with our client or his assistant manually .So the aim for our System is to give solutions to the patients and the employees (receptionist , doctor assistant and the doctor). An easy to use web application the patient can use it to book appointments and contact the doctors's assistant ,the receptionist to view the appointments schedule with no conflicts through the day and for the doctor to view the correct patients' details. We will have a relational database to manage the booking and the patient's history and contacting the Doctor according to the ministry of health laws with an interactive web techniques using HTML,CSS,Bootstrap and PHP . Above all we hope to provide a good user experience with best results. 


\subsection{Overview}
Provides a brief overview of the product defined as a result of the requirements elicitation process. 


\section{General Description}

\subsection{Product Functions}
Describes the general functionality of the product, which will be discussed in more detail below. 

\subsection{Similar System Information}
Describes the relationship of this product with any other products. Specifies if this product is intended to be stand-alone, or else used as a component of a larger product. If the latter, this section discusses the relationship of this product to the larger product. 

\subsection{ User Characteristics}
Describes the features of the user community, including their expected expertise with software systems and the application domain. 

\subsection{ User Problem Statement}
This section describes the essential problem(s) currently confronted by the user community. 

\section{Functional Requirements}
This section lists the functional requirements in ranked order. Functional requirements describes the possible effects of a software system, in other words, what the system must accomplish. Other kinds of requirements (such as interface requirements, performance requirements, or reliability requirements) describe how the system accomplishes its functional requirements. Each functional requirement should be specified in a format similar to the following: 
1.	Short, imperative sentence stating highest ranked functional requirement.
1.	Description
A full description of the requirement. 
2.	Criticality
Describes how essential this requirement is to the overall system. 
3.	Technical issues
Describes any design or implementation issues involved in satisfying this requirement. 
4.	Cost and schedule
Describes the relative or absolute costs associated with this issue. 
5.	Risks
Describes the circumstances under which this requirement might not able to be satisfied, and what actions can be taken to reduce the probability of this occurrence. 
6.	Dependencies with other requirements
Describes interactions with other requirements. 
7.	... others as appropriate
2.	<Name of second highest ranked requirement>
And so forth... 

\section{Interface Requirements}
This section describes how the software interfaces with other software products or users for input or output. Examples of such interfaces include library routines, token streams, shared memory, data streams, and so forth. 

\subsection{User Interfaces}
Use some software for primitive plan of your project.
Describes how this product interfaces with the user. 

\subsubsection {GUI}
Describes the graphical user interface if present. This section should include a set of screen dumps or mockups to illustrate user interface features. 
If the system is menu-driven, a description of all menus and their components should be provided. 

\subsubsection { CLI}
Describes the command-line interface if present. For each command, a description of all arguments and example values and invocations should be provided. 

\subsection {API}
Describes the application programming interface, if present. For each public interface function, the name, arguments, return values, examples of invocation, and interactions with other functions should be provided. 

\section{Design Constraints}
Specifies any constraints for the design team using this document. 

\section{Other non-functional attributes}
Specifies any other particular non-functional attributes required by the system. Examples are provided below. 

\subsection {Security}
\subsection {Reliability}
\subsection {Maintainability}
\subsection {Portability}
\subsection {Extensibility}

\section{Preliminary Object-Oriented Domain Analysis}
This section presents a list of the fundamental objects that must be modeled within the system to satisfy its requirements. The purpose is to provide an alternative, ''structural'' view on the requirements stated above and how they might be satisfied in the system. A primitive class diagram to be delivered.

\subsection{Inheritance Relationships}
This section should contain a set of graphs that illustrate the primary inheritance hierarchy (is-kind-of) for the system. For example: 

\begin{figure}[tbh]
\centering
\includegraphics[width=0.7\linewidth]{./image}
\caption{Inheritance Relations}
\label{fig:image}
\end{figure}

\subsection{Class descriptions}
This section presents a more detailed description of each class identified during the OO Domain Analysis.
Each class description should conform to the following structure: 

\begin{table}[h]
\caption{Class Name - }
\label{tab:my-table}
\begin{tabular}{|p{0.25\textwidth}|p{0.75\textwidth}|}
\hline
\textbf{Abstract or Concrete:} & Indicates whether this class is abstract or concrete.
\\ \hline
\textbf{List of Superclasses}  & Names all immediate superclasses.                                                       
\\ \hline
\textbf{List of Subclasses}    & List of Subclasses                                                                      
\\ \hline
\textbf{Purpose}               & Purpose                                                                                 
\\ \hline
\textbf{Collaborations}        & Names each class with which this class must interact in order to accomplish its purpose, and how.
\\ \hline
\textbf{Attributes}            & Lists each attribute (state variable) associated with each instance of this class, and indicates examples of possible values (or a range).
\\ \hline
\textbf{Operations}            & Lists each operation that can be invoked upon instances of this class.  For each operation, the arguments (and their type), the return value (and its type), and any side effects of the operation should be specified. 
\\ \hline
\textbf{Constraints}           & Lists any restrictions upon the general state or behavior of instances of this class.   
\\ \hline
\end{tabular}
\end{table}

\section{Operational Scenarios}
This section should describe a set of scenarios that illustrate, from the user's perspective, what will be experienced when utilizing the system under various situations. 
In the article Inquiry-Based Requirements Analysis (IEEE Software, March 1994), scenarios are defined as follows: 
In the broad sense, a scenario is simply a proposed specific use of the system. More specifically, a scenario is a description of one or more end-to-end transactions involving the required system and its environment. Scenarios can be documented in different ways, depending up on the level of detail needed. The simplest form is a use case, which consists merely of a short description with a number attached. More detailed forms are called scripts. These are usually represented as tables or diagrams and involved identifying an action and the agent (doer) of the action. FOr this reason, a script can also be called an action table. 
Although scenarios are useful in acquiring and validating requirements, they are not themselves requirements, because the describe the system's behavior only in specific situations; a specification, on the other hand, describes what the system should do in general. 

\section{Project Plan}
This section provides an initial version of the project plan, including the major tasks to be accomplished, their interdependencies, and their tentative start/stop dates. The plan also includes information on hardware, software, and  resource requirements. 
The project plan should be accompanied by one or more PERT or GANTT charts. 

\section{Appendices}
Specifies other useful information for understanding the requirements. All SRS documents should include at least the following two appendices: 

\subsection{Definitions, Acronyms, Abbreviations}
Provides definitions of unfamiliar definitions, terms, and acronyms. 

\subsection{Collected material}

\section {References}

\bibliographystyle{IEEEtranS}

\bibliography{cite}

\end{document}
